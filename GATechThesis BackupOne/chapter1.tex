\chapter{Introduction and Background}


\section{Overview}
Social network analysis SNA is not a formal theory in sociology but rather a strategy for investigating social structures. which uses mathematics to map and quantify the interconnection between actors in a network, an idea which is studied and applied in many fields. In fields like computer networks the aim is to understand the structural relationship between nodes as well as explain why this relationship occur. Moreover, the internet represent a social network of unrivaled scale. This interdisciplinary academic field has many advantages, as it equips researcher and scholars with tools with which to prove theory and study social network structures.  

A principal guiding assumption in Social Network Analysis is that the medium used by  a group to communicated to each other affects some important characteristics of the group. this can be measured on how effective the group are in performing particular tasks\cite{scott2017social}. This is in contrast with the part of social sciences that assumes actors acts and make decision with no regard to the behavior of other actors. 

Studies on off-line social networks have yielded fascinating properties such as the six degrees of separation effect or 'small word'\cite{travers1967small}, high clustering coefficient\cite{newman2004analysis} or scale-free effec\cite{castellsInformationalism}. Others have also showed similar phenomenon in their online social network studies, realizing pattern that are explained by the scale-free effect and small world most notably Biehler and Leskovee\cite{leskovecPlanetary}. The important element considered from Milgram research 1967 is that the majority of members in a social network can be assumed as a single graph. 

In graph theory and network analysis, the concept of centrality was developed by people who sought insight as to how social power is acquired within a group. which they emphasized by saying that an individual does not have power in the abstract, they have power because they can dominate others. Centrality measure help identifies the important node within a graph or network.  This process of identifying important node within a graph have various application which included identifying import infrastructure nodes on the network, network like the internet, it also is used to help identify key spreaders of content as well as studying how information go viral, it also be used to discover various efficient transmission route in a network. 

Centrality measures can  be used to measure popularity in a network, Nodes identified to be popular have high access to others and have a large number of other nodes willing to share information  with them\cite{cross2001beyond}. However high popularity and high influence don't mean the same thing which means that influence does not mean popularity and popularity does not imply influence. 

Four basic measures of centrality have been developed over the years, from which every other sub variant has been derived from, each one exploring the various characteristics that makes a node central in a network, this four measures are:



\section{Aims and Objectives}

With various people using Online social networks (OSN) to communicate all over the world, and with recent growth of the Information centric networking (ISN) paradigm which involves the identification of data based on it name instead of it physical location and relies on in-network caching to store copies of data in the network, we adopt the Content Centric Networking(CCN) architecture of ICN and propose using social information in the design of a new caching strategy for a content centric networking that extends storage capabilities of CDN. We simulate how using centrality can be used to analyze a social network to determine possible location where content can be cached in the network. 
The main research contribution will include, how content storage can effectively be distributed over a CCN Network that extends the storage capability of a CDN. 



\section{Structure of thesis}

This paper is organized as follows, first we outline the literature of Online social network analysis(OSN), the literature of the different centrality measures used to analyses online social networks, exploring their various applications, next we introduce the problem statement and the explain the current limitation of CDN networks and explain how information centric networking and social network analysis can be used to improve the performance of this well adopted networks, following this we introduce the proposed framework we have adopted to go about doing this, then we introduce how social network can be modelled as graphs and we explain how using the ICN paradigm can improve the performance of content storage in a network. Next, we describe how we simulate using Facebook data the principles of content-centric networking and execution of ICN. Finally, we conclude the paper and specify our future work. 



\section{Summary}

Online social network like Facebook, Twitter, YouTube is all the rage this day, this network connects a great deal of people together. The result of the connection and communication of people on this social network continue to influence people personal lives as well as their business. Completely changing how people get and consume data, the availability of popular content on networking platform, people are effortlessly updated on current events. This network of how people communicated has therefore influences how content on the internet tend to be distributed. Popularly still adopted are Content delivery networks(CDNs), which provide user with better experience when accessing contents.  This service uses host to host IP principles to distribute content in the network and fails to consider social semantics when distribution or transferring content to improve content caching. 
Information centric networking(ICN) is a popular new paradigm that has continued to be researched by the community, exploring how content that user require can be stored and delivered to them, replacing the host-host communication required to locate and deliver content user want. Popular content exchanged in social network includes photos or videos posted by individuals. Here we simulate how we can leverage information from this user network to improve storage performance of the network, reducing the latency delay and bandwidth requirement needed. 























