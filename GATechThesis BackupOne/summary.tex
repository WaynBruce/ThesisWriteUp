\clearpage
\begin{centering}
\textbf{SUMMARY}\\
\vspace{\baselineskip}
\end{centering}

A sizable percentage of the population today use online social network application to exchange different form of information and content over the internet. The recent growth of this online social networks has lead to the exploration of new paradigm that explore the content users desire to procure instead of the servers that make this content available, this paradigm is named Information centric networking, this paradigm focuses on a content-based approach instead of the traditional host-based approach. This new paradigm considers online social network behavior i.e. followers of user in popular media subscribe to content posted by the individual they follow, like how people subscribed to YouTube channel get notification of additional content once it is posted in the channel, or like how friends receive content from friend in Facebook. 
Here we recommend the use of Social Network Centrality alongside Content centric networking to help improve user access to content as well as maximize the use of their network resources. We show how Online Social network analyses can be used with Content centric solution, and the simulation we perform highlights the benefit of using this approach, how this approach improves upon the regular Content Deliver Network IP-based classification used in the distribution of content. We use Facebook data, mapping users to a network graph, using the nodes social proximity or centrality to determine where content should be cached, users of the network having access to this data using the relative shortest path will improve the overall network performance and improve the time needed to access content. 


%\pagenumbering{gobble}  %remove page number on summary page